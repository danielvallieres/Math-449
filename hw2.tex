\documentclass[reqno]{amsart} 
\usepackage{amssymb,latexsym,amsmath,amscd,graphicx,setspace,amsthm,verbatim}
\usepackage[margin = 3 cm]{geometry}


\theoremstyle{plain}
\newtheorem{theorem}{Theorem}[section]
\newtheorem{proposition}{Proposition}
\newtheorem{corollary}{Corollary}
\newtheorem{lemma}{Lemma}
\newtheorem{conjecture}{Conjecture}
\newtheorem{question}{Question}
\newtheorem{problem}{Problem}
      
\theoremstyle{definition}
\newtheorem{definition}{Definition}

\newenvironment{solution}{\paragraph{\emph{Solution}.}}{\hfill$\square$}


\newenvironment{solution1}{\paragraph{\emph{Solution $1$}.}}{\hfill$\square$}
\newenvironment{solution2}{\paragraph{\emph{Solution $2$}.}}{\hfill$\square$}
\newenvironment{solution3}{\paragraph{\emph{Solution $3$}.}}{\hfill$\square$}

\begin{document} 

\title[Homework 2]{Homework 2 \\ First due date: None.  \\  Due date for the second submission:  None.}

\date{\today} 
\maketitle 


\begin{problem}
Let $X$ be the points on the unit circle in the euclidean plane $\mathbb{E}^{2}$, i.e.
$$X = \{(x,y) \in \mathbb{R}^{2} \, | \, x^{2} + y^{2} = 1 \}. $$
Define a binary operation $\oplus$ on $X$ via
$$(x_{1},y_{1}) \oplus (x_{2},y_{2}) = (x_{1}y_{2} + x_{2}y_{1},y_{1}y_{2} - x_{1}x_{2}). $$
Show that $(X,\oplus)$ is an abelian group.
\end{problem}
\begin{proof}

\end{proof}


\begin{problem}
Let $(G,*)$ be a group and let $a \in G$.  Show that $a$ is the neutral element of $G$ if and only if $a * a = a$.
\end{problem}
\begin{proof}

\end{proof}

\begin{problem}
Assume that $X$ is a finite set with a binary operation $*:X \times X \longrightarrow X$ satisfying the first two group axioms, i.e. associativity and the existence of a neutral element.  Assume every $a \in X$ has a left inverse, say $b$, and a right inverse, say $c$.  This means:
\begin{enumerate}
\item $b * a = e$
\item $a * c = e$.
\end{enumerate}
Show that $b = c$ and thus deduce that $(X,*)$ is a group.
\end{problem}
\begin{proof}

\end{proof}

\begin{problem}
Consider the group $S_{4}$ with its usual binary operation, namely composition of functions (also denoted by $\cdot$ in this case instead of the usual $\circ$).  Let
$$\tau_{1} = \left( \begin{smallmatrix} 1 & 2 & 3 & 4 \\ 2 & 4 & 3 & 1 \end{smallmatrix}\right), \tau_{2} =  \left( \begin{smallmatrix} 1 & 2 & 3 & 4 \\ 3 & 1 & 2 & 4 \end{smallmatrix}\right)$$
\begin{enumerate}
\item Calculate $\tau_{1} \cdot \tau_{2}$
\item Calculate $(\tau_{1} \cdot \tau_{2}) \cdot \tau_{2}$
\item Calculate $\tau_{1} \cdot \tau_{1} \cdot \tau_{1} \cdot \tau_{1}$
\end{enumerate}
\end{problem}
\begin{solution}

\end{solution}







\end{document} 



