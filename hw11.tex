\documentclass[reqno]{amsart} 
\usepackage{amssymb,latexsym,amsmath,amscd,graphicx,setspace,amsthm,verbatim}
\usepackage[margin = 3 cm]{geometry}


\theoremstyle{plain}
\newtheorem{theorem}{Theorem}[section]
\newtheorem{proposition}{Proposition}
\newtheorem{corollary}{Corollary}
\newtheorem{lemma}{Lemma}
\newtheorem{conjecture}{Conjecture}
\newtheorem{question}{Question}
\newtheorem{problem}{Problem}
      
\theoremstyle{definition}
\newtheorem{definition}{Definition}

\newenvironment{solution}{\paragraph{\emph{Solution}.}}{\hfill$\square$}


\newenvironment{solution1}{\paragraph{\emph{Solution $1$}.}}{\hfill$\square$}
\newenvironment{solution2}{\paragraph{\emph{Solution $2$}.}}{\hfill$\square$}
\newenvironment{solution3}{\paragraph{\emph{Solution $3$}.}}{\hfill$\square$}

\begin{document} 

\title[Homework 11]{Homework 11}

\date{\today} 
\maketitle 


\begin{problem}
Consider the group $\mathbb{Z}/4\mathbb{Z}$ and the bijection $f:\mathbb{Z}/4\mathbb{Z} \rightarrow \{1,2,3,4 \}$ given by
$$\bar{0}\mapsto 3, \bar{1} \mapsto 1, \bar{2} \mapsto 2, \bar{3} \mapsto 4. $$
Consider also the composition $G \hookrightarrow {\rm Sym}(G) \stackrel{\tilde{f}}{\longrightarrow} S_{4}$ as we did in class when discussing Cayley's theorem.  Via this composition, find the subgroup of $S_{4}$ that is isomorphic to $G$.  (In other words, find the image of $G$ under this previous composition.)
\end{problem}
\begin{solution}

\end{solution}

\begin{problem}
Consider the group $S_{3}$ and the bijection $f:S_{3} \rightarrow \{1,2,3,4,5,6 \}$ given by
$$1_{S_{3}}\mapsto 6,(1 \, 2) \mapsto 1, (1 \, 3) \mapsto 2, (2 \, 3) \mapsto 3, (1 \, 2 \, 3) \mapsto 4, (1 \, 3 \, 2) \mapsto 5. $$
Repeat the same exercise as in the previous problem.
\end{problem}
\begin{solution}

\end{solution}

For the remaining problems, you will start thinking about the notions of \emph{conjugation} and of \emph{normal subgroups}.  For that purpose, let us introduce the following definition:  If $G$ is a group and $S$ is any subset of $G$, then for every $g \in G$, we define the subset
$$gSg^{-1} = \{gsg^{-1} : s \in S \} \subseteq G.$$



\begin{problem}
\hspace{1cm}
\begin{enumerate}
\item Let $G$ be a group and let $H$ be a subgroup of $G$.  Show that for all $g \in G$, one has $gHg^{-1} \le G$.  (The subgroup $gHg^{-1}$ is called a \emph{conjugate subgroup} of $H$.)
\item Let $G = S_{3}$, $H = \langle (1 \, 2) \rangle$, and let $g = (1 \, 2 \, 3)$.  What is the subgroup $gHg^{-1}$?  Go ahead and play with this some more, meaning calculate a few more examples of conjugate subgroups in your favorite groups.
\end{enumerate}
\end{problem}
\begin{solution}

\end{solution}

For the following problems, we need the following important definition.
\begin{definition}
Let $G$ be a group, and let $H \le G$.  The subgroup $H$ is called a \emph{normal subgroup} of $G$, and one writes $H \unlhd G$, if for all $g \in G$, one has
$$gHg^{-1} \subseteq H. $$
\end{definition}

\begin{problem}
\hspace{1cm}
\begin{enumerate}
\item Let $G = S_{3}$, and $H = \langle (1 \, 2) \rangle$?  Is $H \unlhd G$?  
\item Can you find a subgroup of $S_{3}$ that is normal?
\end{enumerate}
\end{problem}
\begin{solution}

\end{solution}

\begin{problem}
Let $f:G_{1} \rightarrow G_{2}$ be a group homomorphism between two groups.  Show that ${\rm ker}(f) \unlhd G_{1}$.
\end{problem}
\begin{solution}

\end{solution}









































\end{document} 

