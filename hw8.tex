\documentclass[reqno]{amsart} 
\usepackage{amssymb,latexsym,amsmath,amscd,graphicx,setspace,amsthm,verbatim}
\usepackage[margin = 3 cm]{geometry}


\theoremstyle{plain}
\newtheorem{theorem}{Theorem}[section]
\newtheorem{proposition}{Proposition}
\newtheorem{corollary}{Corollary}
\newtheorem{lemma}{Lemma}
\newtheorem{conjecture}{Conjecture}
\newtheorem{question}{Question}
\newtheorem{problem}{Problem}
      
\theoremstyle{definition}
\newtheorem{definition}{Definition}

\newenvironment{solution}{\paragraph{\emph{Solution}.}}{\hfill$\square$}


\newenvironment{solution1}{\paragraph{\emph{Solution $1$}.}}{\hfill$\square$}
\newenvironment{solution2}{\paragraph{\emph{Solution $2$}.}}{\hfill$\square$}
\newenvironment{solution3}{\paragraph{\emph{Solution $3$}.}}{\hfill$\square$}

\begin{document} 

\title[Homework 8]{Homework 8}

\date{\today} 
\maketitle 

\begin{problem}
Let $G$ be a group and let $a, b \in G$.  Assume that $a$ and $b$ commute, that is $ab = ba$.  
\begin{enumerate}
\item Prove that if $a^{m} = 1 = b^{n}$ for some integers $m$ and $n$, then $(ab)^{k} = 1$, where $k$ is any common multiple of $m$ and $n$.
\item With the same setup as above, if $m ={\rm Ord}(a)$ and $n = {\rm Ord}(b)$, then does one have that the order of $ab$ is the least common multiple of $m$ and $n$?  If yes, prove this, otherwise give a counterexample.
\item Let $G = {\rm GL}(2,\mathbb{R})$ and consider the matrices
\begin{equation*}
A =
\begin{pmatrix}
0 & -1 \\ 1 & 0
\end{pmatrix}
\text{ and }
B=
\begin{pmatrix}
0 & 1 \\ -1 & -1
\end{pmatrix}
\end{equation*}
Find the order of $A$, $B$ and $AB$.
\item Do $A$ and $B$ commute in $G$?
\end{enumerate}
\end{problem}

\begin{solution}

\end{solution}

\begin{problem}
Let $G$ be a group and define the subset
$$G_{{\rm tor}} = \{g \in G \, | \, {\rm Ord}(g) < \infty \}. $$
\begin{enumerate}
\item Show that if $G$ is abelian, then $G_{{\rm tor}} \le G$.  (This subgroup is called the torsion subgroup of $G$.)
\item Is this statement still true if one drops the hypothesis that $G$ is abelian?  (Hint:  Think about Problem $1$.)
\end{enumerate}
\end{problem}

\begin{solution}

\end{solution}

\begin{problem}
Let $G_{1}$ and $G_{2}$ be two groups and let $f:G_{1} \longrightarrow G_{2}$ be a group morphism.  Show that $f$ is injective if and only if
$${\rm ker}(f) = \{1_{G_{1}}\}. $$
\end{problem}
\begin{proof}

\end{proof}

\begin{problem}
Consider the function $f:\mathbb{C}^{\times} \longrightarrow {\rm GL}(2,\mathbb{R})$ defined via
\begin{equation*}
z = a + bi \mapsto f(z) = 
\begin{pmatrix}
a & -b \\
b & a
\end{pmatrix}.
\end{equation*}
Show that $f$ is an injective group homomorphism.
\end{problem}
\begin{proof}

\end{proof}


For the next problems, you want to review the cartesian product of two groups $G_{1} \times G_{2}$ defined in a previous hw.
\begin{problem}
\hspace{1cm}
\begin{enumerate}
\item What is the cardinality of $\mathbb{Z}/2\mathbb{Z}\times\mathbb{Z}/2\mathbb{Z}$ and of $\mathbb{Z}/4\mathbb{Z}$?  (The group $\mathbb{Z}/2\mathbb{Z} \times \mathbb{Z}/2\mathbb{Z}$ is called the Klein four group.)
\item What are the orders of the elements in $\mathbb{Z}/2\mathbb{Z}\times \mathbb{Z}/2\mathbb{Z}$?  
\item What are the orders of the elements in $\mathbb{Z}/4\mathbb{Z}$?
\item Do we have $\mathbb{Z}/2\mathbb{Z} \times \mathbb{Z}/2\mathbb{Z}\simeq \mathbb{Z}/4\mathbb{Z}$?  (Hint:  Problem 3 of hw 5 might be handy...)
\end{enumerate}
\end{problem}
\begin{proof}

\end{proof}

\begin{problem}
Let $G_{1}$ and $G_{2}$ be two groups and consider the cartesian product $G_{1} \times G_{2}$.  For $i=1,2$, define the function
$$\pi_{i}: G_{1} \times G_{2} \longrightarrow G_{i} $$
via $(g_{1},g_{2}) \mapsto \pi_{i}(g_{1},g_{2}) = g_{i}$.
\begin{enumerate}
\item Show that $\pi_{i}$ is a group morphism.
\item Is $\pi_{i}$ surjective?
\item What is ${\rm ker}(\pi_{i})$?
\end{enumerate}
\end{problem}
\begin{proof}

\end{proof}






































\end{document} 

