\documentclass[reqno]{amsart} 
\usepackage{amssymb,latexsym,amsmath,amscd,graphicx,setspace,amsthm,verbatim}
\usepackage[margin = 3 cm]{geometry}


\theoremstyle{plain}
\newtheorem{theorem}{Theorem}[section]
\newtheorem{proposition}{Proposition}
\newtheorem{corollary}{Corollary}
\newtheorem{lemma}{Lemma}
\newtheorem{conjecture}{Conjecture}
\newtheorem{question}{Question}
\newtheorem{problem}{Problem}
      
\theoremstyle{definition}
\newtheorem{definition}{Definition}

\newenvironment{solution}{\paragraph{\emph{Solution}.}}{\hfill$\square$}


\newenvironment{solution1}{\paragraph{\emph{Solution $1$}.}}{\hfill$\square$}
\newenvironment{solution2}{\paragraph{\emph{Solution $2$}.}}{\hfill$\square$}
\newenvironment{solution3}{\paragraph{\emph{Solution $3$}.}}{\hfill$\square$}

\begin{document} 

\title[Homework 5]{Homework 5 \\ First due date: October 10.  \\  Due date for the second submission:  TBD.}

\date{\today} 
\maketitle 



\begin{problem}
Consider the group $\mathbb{R}$ with its usual binary operation, namely addition, and also the group ${\rm GL}(2,\mathbb{R})$ with its usual binary operation, namely multiplication of matrices.  Consider the function $f:\mathbb{R} \longrightarrow {\rm GL}(2,\mathbb{R})$ defined via
$$\theta \mapsto f(\theta) =\begin{pmatrix} \cos \theta & -\sin \theta \\ \sin \theta & \cos \theta \end{pmatrix}. $$
\begin{enumerate}
\item Show that $f$ is a group morphism.
\item Is $f$ surjective?
\item Is $f$ injective?
\item Is $f$ and isomorphism?
\end{enumerate}
\end{problem}
\begin{solution}

\end{solution}

\begin{problem}
Let $X$ be the points on the unit circle in $\mathbb{E}^{2}$, i.e.
$$X = \{(x,y) \in \mathbb{E}^{2} \, | \, x^{2} + y^{2} = 1 \}. $$
Define a binary operation $\oplus$ on $X$ via
$$(x_{1},y_{1}) \oplus (x_{2},y_{2}) = (x_{1}y_{2} + x_{2}y_{1},y_{1}y_{2} - x_{1}x_{2}). $$
Recall that you showed in a previous hw assignment that $(X,\oplus)$ is an abelian group.
\begin{enumerate}
\item Show that the function $f:\mathbb{R} \longrightarrow X$ defined via
$$\theta \mapsto f(\theta) = (\sin\theta,\cos \theta) $$
is a group morphism.
\item Is $f$ surjective?
\item Is $f$ injective?
\item Is $f$ an isomorphism?
\end{enumerate}
\end{problem}
\begin{solution}

\end{solution}


\begin{problem}
Let $G_{1}$ and $G_{2}$ be two groups and assume that $f:G_{1} \longrightarrow G_{2}$ is a group morphism.
\begin{enumerate}
\item Show that for all $g_{1} \in G_{1}$ such that ${\rm Ord}(g_{1})$ is finite, one has ${\rm Ord}(f(g_{1})) \, | \, {\rm Ord}(g_{1})$.
\item Assume furthermore that $f$ is a group isomorphism.  Show that for all $g_{1} \in G_{1}$, one has ${\rm Ord}(f(g_{1})) = {\rm Ord}(g_{1})$.
\end{enumerate}  
\end{problem}
\begin{proof}

\end{proof}

The following problem is an important construction of a new group out of two known groups.  It is called the cartesian product of the two groups $G_{1}$ and $G_{2}$.
\begin{problem}
Let $G_{1}$ and $G_{2}$ be two groups and for clarity, let us denote the binary operation of $G_{1}$ by $*$ and the binary operation of $G_{2}$ by $\#$.  Consider the cartesian product
$$G_{1} \times G_{2} = \{(g_{1},g_{2}) \, | \, g_{i} \in G_{i} \text{ for } i=1,2  \}. $$
Show that $G_{1} \times G_{2}$ endowed with the binary operation $\cdot$ defined via
$$(g_{1},g_{2}) \cdot (g_{1}',g_{2}') = (g_{1} * g_{1}', g_{2} \# g_{2}') $$
is a group.
\end{problem}
\begin{proof}

\end{proof}

Make sure you understand the following concepts from Math 330:  \emph{Equivalence relation}, \emph{equivalence class}, \emph{representative}, \emph{partition}.  If you are a little shaky on this matter, pick up your favorite book on basic set theory, then read, learn, and exercise.

\begin{problem}
Consider the relation $\sim$ on $\mathbb{R}$ defined via $x \sim y$ if $|x| = |y|$.  Answer the following questions:
\begin{enumerate}
\item Is $\sim$ an equivalence relation on $\mathbb{R}$?  If yes, prove this, otherwise explain why not.
\item What is $[-1]$, i.e. what is the equivalence class of $-1$.
\item Can you give me another representative for the equivalence class $[-1]$?
\item How many distinct equivalence classes are there?
\item What is the set $[2] \cap [3]$?
\item What about the set $[2] \cap [-2]$?
\item What is the cardinality of each equivalence class?
\end{enumerate}
\end{problem}
\begin{solution}

\end{solution}










\end{document} 



